\documentclass{notas}
\title{Teoria de Galois finita}
\setdefaultlanguage{spanish}%[variant=mexican]
\begin{document}

\maketitle

\chapter{Extensiones de campo y el teorema fundamental.}
\section{Tipos de extensiones de campo.}

Para comenzar con las nociones principales debemos partir de la noción de extensión de campo.

\begin{tcolorbox}[title=Definición]
	Sean $E,F$ campos. Decimos que $E$ es extensión de campo de $F$ si $F$ es \textit{subcampo} de $E$, denotado por: $$E/F$$
\end{tcolorbox}

Podemos pensar las extensiones de campo como \textit{espacios vectoriales}, es decir, si $E/F$ es extensión de campo entonces $E$ es un $F$-espacio vectorial vía la \textbf{acción de multiplicación}.
En este sentido, podemos hablar de la \textit{dimensión} de este espacio vectorial

\begin{tcolorbox}[title=Grado de una extensión]
	Sea $E/F$ extensión de campos. Definimos el \underline{grado de la extensión} como:$$[E:F]=\dim_F(E)$$
\end{tcolorbox}

De la teoría de modulos y que los campos son anillos de división conmutativos, tenemos que el grado de una extensión es \textbf{multiplicativo}:

\begin{tcolorbox}[title=Propiedad multiplicativa del grado]
	Sea una \textit{torre de campos}: $K\subseteq F\subseteq E$. Entonces: $$[E:K]=[E:F][F:K]$$ 
\end{tcolorbox}

\end{document}
